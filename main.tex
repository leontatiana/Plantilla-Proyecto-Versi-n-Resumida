%--------------------------------IMPORTANTE-------------------------------
%-------------------------------------------------------------------------
%	1. Se debe modificar aquello que este comentado de la siguiente manera
							%"~~~~~MODIFICAR~~~~~~~~"
%	2. Sólo eliminar y reemplazar aquello que diga explicitamente
							%"eliminar"
%	3. Leer MUY BIEN los comentarios

%Esta plantilla fue desarrollada por Tatiana León Zamora, cualquier duda, pueden contactarme: yessica.leon@mail.escuelaing.edu.co
%-------------------------------------------------------------------------
%-------------------------------------------------------------------------
\documentclass[12pt, oneside]{article}
\usepackage[utf8]{inputenc}
\usepackage[spanish]{babel}
%Los siguientes son los packages más básicos para la escritura del artículo
\usepackage{graphicx, subfigure, wrapfig, float}
\usepackage{amsmath, amssymb, amsfonts, amsthm}
\usepackage{caption, multicol, multirow, longtable}
\usepackage{url, cclicenses, lastpage, color, booktabs}
\usepackage{graphicx,lipsum,wrapfig} %lipsum es un generador de texto dummy
\usepackage{fancyhdr, fancybox}
\usepackage{array, cite, enumerate} 
\usepackage{vmargin}
\usepackage{anysize}
\usepackage{xcolor}

%Define comandos para cuando se quiera poner un teorema, lema, ejemplo, corolario o definición
\newtheorem{theo}{Teorema}
\newtheorem{examp}{Ejemplo}
\newtheorem{coro}{Corolario}
\newtheorem{defi}{Definición}
\newtheorem{lem}{Lema}


\renewcommand\rmdefault{lmr}

\cfoot{\thepage} %pie de página

\begin{document}
\setmargins{2cm}{1cm}{18cm}{23.42cm}{10pt}{1cm}{0pt}{1cm}

%encabezado portada
\begin{figure}[t]
\begin{minipage}{0.5\textwidth}\large
\begin{flushleft}
\includegraphics[scale=0.7]{logoECI.png} %encabezado ECI
\end{flushleft}
\end{minipage}
\begin{minipage}{0.5\textwidth}\large
\begin{flushright}
\includegraphics[scale=0.75]{MATH.png} %encabezado Programa de Matemáticas
\end{flushright}
\end{minipage}
\end{figure}

%~~~~~~~~~~~~~~~~~~~~~~~~~~~~MODIFICAR~~~~~~~~~~~~~~~~~~~~~
\title{\textsc{Ponga aquí el título de su proyecto}} %Poner el título de su proyecto
%~~~~~~~~~~~~~~~~~~~~~~~~~~~~~~~~~~~~~~~~~~~~
\date{}
\maketitle
\begin{center}
\begin{tabular}{cc}
\textbf{Autores} & \textbf{Dirigido Por} \\
%~~~~~~~~~~~~~~~~~~~~~~~~~~~~MODIFICAR~~~~~~~~~~~~~~~~~~~~~
%Poner el nombre de los autores sustituyendo Nombre Apellido 1, Nombre Apellido 2, Nombre Apellido 3. Añadir o eliminar filas si es necesario

Nombre Apellido 1 & PhD./MSc./Profesor. Nombre Profesor\\ %Poner si el profe es PhD., MSc. o Profesor.
Nombre Apellido 2  &  \\
Nombre Apellido 3 & \\		
%~~~~~~~~~~~~~~~~~~~~~~~~~~~~~~~~~~~~~~~~~~~~
\end{tabular}
\end{center}
\vspace{1cm}

\textbf{Resumen}
\\ \\
%~~~~~~~~~~~~~~~~~~~~~~~~~~~~MODIFICAR~~~~~~~~~~~~~~~~~~~~~
%eliminar \lipsum[#]
Debe ser un pequeño resumen de lo que trata su trabjo.
\\ \\
\lipsum[1]
%~~~~~~~~~~~~~~~~~~~~~~~~~~~~~~~~~~~~~~~~~~~~
\vspace{1cm}

%~~~~~~~~~~~~~~~~~~~~~~~~~~~~MODIFICAR~~~~~~~~~~~~~~~~~~~~~
\textbf{Palabras Clave}
\\ \\
%~~~~~~~~~~~~~~~~~~~~~~~~~~~~MODIFICAR~~~~~~~~~~~~~~~~~~~~~
Phasellus, donec, eget, trum \\ \\ %eliminar
%~~~~~~~~~~~~~~~~~~~~~~~~~~~~~~~~~~~~~~~~~~~~

\vspace{1cm}
%~~~~~~~~~~~~~~~~~~~~~~~~~~~~MODIFICAR~~~~~~~~~~~~~~~~~~~~~

\textbf{Abstract}
\\ \\
%eliminar \lipsum[#]
It should be a brief summary of what your project is about.
\\ \\
\lipsum[1]
%~~~~~~~~~~~~~~~~~~~~~~~~~~~~~~~~~~~~~~~~~~~~
\vspace{1cm}

\textbf{Keywords}
\\ \\
%~~~~~~~~~~~~~~~~~~~~~~~~~~~~MODIFICAR~~~~~~~~~~~~~~~~~~~~~
Phasellus, donec, eget, trum \\ \\ %eliminar 
%~~~~~~~~~~~~~~~~~~~~~~~~~~~~~~~~~~~~~~~~~~~~
\vspace{1cm}
\textbf{Preliminares}
\\ \\
%~~~~~~~~~~~~~~~~~~~~~~~~~~~~MODIFICAR~~~~~~~~~~~~~~~~~~~~~
%eliminar \lipsum[#]
\lipsum[1]
%~~~~~~~~~~~~~~~~~~~~~~~~~~~~~~~~~~~~~~~~~~~~
\vspace{1cm}

\textbf{Resultados Importantes} \\
\\
%~~~~~~~~~~~~~~~~~~~~~~~~~~~~MODIFICAR~~~~~~~~~~~~~~~~~~~~~
%eliminar \lipsum[#]
Se debe poner algunos resultados importantes, o ejemplos que ilustren de lo que trata su trabajo. Es importante tener en cuenta que no se debe poner TODOS los resultados del trabajo principal. Sólo los más importantes.
%~~~~~~~~~~~~~~~~~~~~~~~~~~~~~~~~~~~~~~~~~~~~
\\ \\
\lipsum[3]
\vspace{1cm}
\textbf{Conclusiones}
\\ \\
%~~~~~~~~~~~~~~~~~~~~~~~~~~~~MODIFICAR~~~~~~~~~~~~~~~~~~~~~
%eliminar \lipsum[#]
\lipsum[4]
%~~~~~~~~~~~~~~~~~~~~~~~~~~~~~~~~~~~~~~~~~~~~
\vspace{1cm}
\textbf{Trabajo a Futuro (si lo hay)}
\\ \\
%~~~~~~~~~~~~~~~~~~~~~~~~~~~~MODIFICAR~~~~~~~~~~~~~~~~~~~~~
%eliminar \lipsum[#]
\lipsum[4]
%~~~~~~~~~~~~~~~~~~~~~~~~~~~~~~~~~~~~~~~~~~~~
\vspace{1cm}
\textbf{Referencias} %Los siguientes son algunos ejemplos de cita.
\begin{enumerate}
%~~~~~~~~~~~~~~~~~~~~~~~~~~~~MODIFICAR~~~~~~~~~~~~~~~~~~~~~
\item LÓPEZ, F. (SF) \textit{Espacios Topoógicos}, Depto. de Geometría y Topología.
\item MACHO, M. (2002), \textit{¿Qué es la topología?}, Universidad del País Vasco-Euskal Herriko Unibertsitatea.
%~~~~~~~~~~~~~~~~~~~~~~~~~~~~~~~~~~~~~~~~~~~~
\end{enumerate}
\vspace{1cm}
\noindent\fbox{%
\parbox{16cm}{%
\textbf{Contactos:}
\\ \\
%~~~~~~~~~~~~~~~~~~~~~~~~~~~~MODIFICAR~~~~~~~~~~~~~~~~~~~~~
Nombre Profesor: profesor@escuelaing.edu.co \\ \\
Nombre Apellido 1: estudiante1@mail.escuelaing.edu.co\\
Nombre Apellido 2: estudiante2@mail.escuelaing.edu.co\\
Nombre Apellido 3: estudiante3@mail.escuelaing.edu.co\\
%~~~~~~~~~~~~~~~~~~~~~~~~~~~~~~~~~~~~~~~~~~~~
}%
}
\\ \\ \\

\end{document}

%Esta plantilla fue desarrollada por Tatiana León Zamora, cualquier duda, pueden contactarme: yessica.leon@mail.escuelaing.edu.co

